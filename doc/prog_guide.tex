\documentclass{report}
\usepackage[utf8]{inputenc}
\usepackage{underscore}

\title{Map My Garage Sale: Programmer's Guide}
\author{Blake Lowe, Jordan Polaniec}

\begin{document}

\pagenumbering{gobble}
\maketitle
\newpage
\pagenumbering{arabic}

\tableofcontents
\newpage

\chapter{Engine}

\section{Tiles \& Grids}
Tiles and Grids are the fundamental data types upon which the rest of the
program is based. Grids are used to represent the Main Grid shown to the user,
and they are also used to represent the shape of a Stand.

Each Grid is made up of many Tiles. Each Tile represents a ``square'' on a
Grid. Tiles contain pointers to their immediately adjacent Tiles on that Grid.
The Grid, therefore, can be seen as a graph consisting of nodes which form
a rectangular matrix.

Grids keep a record of their length and width in Tiles so that their
sizes can be easily queried later.

The Grid structure keeps a pointer to its origin Tile, located in the
top-left corner of the Grid. From this Tile, you can traverse the Grid
using the links between Tiles. If the top-left Tile becomes a different Tile
(for example, due to a rotation), you should call \texttt{reset_origin}, which will
reset the origin pointer to the new location of the top-left Tile.

When referring to a specific Tile in a grid, the column of the Tile
is the number of columns to the right of the origin Tile it is, and
its row is the number of rows below the origin Tile it is.
The coordinates should be expressed as [row, column].
For example, in the following Grid:
\begin{verbatim}
A B C
D E F
G H I
\end{verbatim}
\begin{itemize}
	\item A is the origin Tile, it is at [0, 0].
	\item C is at [0, 2].
	\item H is at [2, 1].
\end{itemize}

Tiles and Grids keep all of their data on the heap. Constructors\footnote{
	\texttt{new_grid} and \texttt{clone_grid} for constructing Grids
	and \texttt{new_tile} for constructing Tiles}
and destructors\footnote{\texttt{del_grid}; Tiles can be free'd} are provided
to create and destroy these types.

To aid in access to a Grid, and to provide an easy way of iterating over all
of a Grid's Tiles, a lookup table is provided. The lookup pointer in a Grid is
an array of pointers to Tiles, which contains a pointers to each Tile in the
Grid in row-major order. To use this feature, the function \texttt{grid_lookup}
is provided, which performs the calculation necessary to find a specific Tile
given its row and column coordinates. If you perform a mutation upon the Grid
that changes the structure of its Tiles (again, rotation being a good example),
you should call \texttt{rebuild_lookup} (after ensuring the origin pointer
is correctly set), which will traverse the Grid, starting from its origin, and
recreate the lookup table.

Reverse lookups are straightforward: each Tile contains the row and column
coordinates at which it resides on its Grid. These coordinates are also
updated during a call to \texttt{rebuild_lookup}.

The currently provided mutation routines are \texttt{rotate_grid}, which
flips it 90 degrees, and \texttt{mirror_grid}, which flips the Grid structure
by moving columns of Tiles to their symmetrical opposite position.
While \texttt{mirror_grid} only mirrors a Grid along the Y axis, an X-axis
mirror can of course be synthesized easily by following a call to
\texttt{mirror_grid} with two calls to \texttt{rotate_grid}
in the same direction.

\section{Stands}
Stands are the other major data type in the program. Stands represent physical
item displays in the real world. They occupy space on Grids by applying
themselves onto Tiles.

These are the structures manupulated on the Main Grid by the user. They occupy
space on Grids by applying themselves onto Tiles. They also carry color
information, which is used to draw them.

When examining a Grid, each Stand can be uniqely identified by its
pointer. Stands may not overlap one another.

Stands are constructed from pre-existing Stand Templates. Stand
Templates include a small Grid over which its shape is applied. This
Grid is only large enough to serve as the minimum bounding-
box of its shape, plus enough added rows and columns to make the template
a square. In addition, new Stand Templates can be defined by
scanning a Grid for a Stand, and recreating it upon a new Grid. Again,
the new Grid must be the minimum boundung box of its shape.

This special Grid, called the ``source'' grid, carries the structure of the
stand so that it can be applied onto a Grid by copying the layout of the source.
The source grid should be conisdered freely mutable; many stand mutation
methods operate by performing the change on the source Grid, then checking
to see if the new variant can be re-applied to the owning Grid.

To apply a stand, one must call two functions: \texttt{can_apply} and
(if true) \texttt{do_apply}. This allows the engine to return information
about the applicability of the stand, but remember information about its
applicability, so that it doesn't have to recreate it during the second call.
The frontend makes calls to these to apply Stands to the Grid. In addition,
many of the Stand mutation methods call these functions after performing a
modification to the Stand's source Grid.

Tiles can contain a pointer to either a stand or a stand_template. This is
achieved via a union type, defined in \texttt{grid.h}. It looks as follows:
\begin{verbatim}
enum stand_type {STAND, STAND_TEMPLATE};
typedef union stand_like {
	struct {
		enum stand_type type;
	} stand_proto;
	struct {
		enum stand_type type;
		stand s;
	} stand_stand;
	struct {
		enum stand_type type;
		stand_template st;
	} stand_st;
} stand_like;
\end{verbatim}
Thus, a function operating on Tiles has a type-safe way of recognizing what
kind of Stand object it holds and can deal with it accordingly. 

Stands are owned by a Grid; it is illegal for Tiles on different Grids to
have references to the same Stand instance at any given time (instead you
should make a new Stand from the original Stand Template).

Stands and Stand Templates exist on the heap, and must be destroyed
to aviod leaking them.

\section{Saving/Loading}
The ability to save and load the program state is an essential part of
any modern program. It is important, therefore, that the processes of
marshalling and unmarshalling the data be performed correctly.

In the engine, saving and loading are performed with the functions
\texttt{save_file} and \texttt{load_file}.

\subsection{File Format}
We utilize an extensible, textual format to represent program data.
It is divided into several logical blocks which each represent a different
component of the program. These blocks may appear in any order. The file
is expected to be encoded in ASCII or UTF-8.

The save file shall begin with the characters ``MMGS'', followed by a colon,
then followed by the file version number, follwed by a number. The
file version number, as well as all other numbers in the file, shall be a
decimal integer spelled out using numeric characters.

The different blocks are as follows:
\begin{itemize}
	\item The \emph{standtemplates} block begins with one square-bracket
		enclosed parameter, a number telling the parser how many
		Stand Templates are expected within the block. It is an error
		for this number to not match the number of definitions within
		the block. After the close bracket shall follow an open
		parenthesis.

		Each definition will begin with a number giving the number of
		bytes in the name of the template\footnote{This is \emph{not}
			the same as the number of logical characters in the
			name. This allows for names with multi-byte characters
		to be properly encoded.}, followed by a colon,
		followed by the name itself. The name shall be followed by
		a colon, followed by three colon-separated integers 
		defining the color of the stand in HTML RGB notation.
		Naturally, these numbers shall be within the range 0--255.
		After these shall follow a colon, then two colon-separated
		numbers defining the height and width, respectively, of
		the Grid which represents the shape of the Stand Template. This
		shall be followed with a colon, a Grid Definition (see below),
		and finally terminated with a semicolon. This semicolon ends
		the Stand Template definition.

		The block shall be ended with a close parenthesis.
	\item The \emph{stands} block begins with the same parameter format as
		the standtemplates block, and it serves the same purpose:
		to inform the parser of the number of Stand definitions to
		expect.

		The format of Stand definitions inside the block are
		identical to the definitions of Stand Templates, with the
		following modification:
		\begin{itemize}
			\item After the Grid Definition shall follow not a
				semicolon, but a colon. After this shall
				follow the row and column, respectively, of
				the location of this Stand's placement
				on the Main Grid.
		\end{itemize}
	\item The \emph{maingrid} block has no parameters, and is opened
		and closed with parenthesis. Inside of the
		block, two colon-separated integers are expected: the
		height and width, respectively, of the Main Grid.

		This block notably does not contain a Grid Definition. The
		makeup of the Main Grid is deduced from the coordinates of
\end{itemize}

The \emph{Grid Definition} is a loose block of the characters ``0'',
representing an empty Tile, and ``S'', representing a filled Tile.
These characters shall appear in row-major order, but whitespace is ignored.
It is recommended, but not required, that the program place spaces between
characters, and newlines at the end of each row of Tiles. This is expected to
improve readability.
\newpage
The following is an example of a fully-featured save file:
\begin{verbatim}
MMGS:1;

standtemplates[2](
7:L Block:0:255:0:255:4:4:
0 S 0 0
0 S 0 0
0 S S 0
0 0 0 0;

6:Square:255:0:0:255:2:2:
S S
S S;
)

stands[2](
5:Table:130:130:130:4:4:
0 0 0
S S S
S S S:
3:2;

4:Case:100:100:100:1:1:
S:
14:13;
)

maingrid(
100:100
)
\end{verbatim}

A conforming parser of MMGS files should ignore whitespace outside of the
names of Stands and Stand Templates.

\chapter{Frontend}

\section{Design}

Map My Garage Sales's User Interface was designed to make core
functionality accessible to the user. Ideally, the user will be able to 
create a new stand in under 10 clicks, allowing the user to create an 
effective map of their garage sale with minimal overhead. A visible grid can be toggled to help
map out Stands as necessary.
\subsection{Stand Placement \& Editing}
Our workflow will move the user to the Create New Stand button to create a new 
Stand and give it a Name, Width, Height and Color and Shape. This Stand will then be placed
in the list of available Stands for them to drag into the Map Area. Each time they 
drag a Stand Template from the list of available Stands a new instance of that Stand 
will be created and shown on the Map Area where they dropped it.

\emph{Rotate Selected Stand} - Clicking this button will rotate a specific instance of
a Stand they have selected in the Map Area by clicking the Rotate Stand button while 
it is selected. This will rotate the stand 90 degrees clockwise for each time they click 
the Rotate Stand button.

\begin{itemize}
	\item NOTE - This will not rotate all Stands of this Stand Template type. Only the
currently selected Stand will be rotated.
\end{itemize}

\emph{Rename Stand} - Clicking the Rename Stand button while a Stand Template is 
selected in the list of available Stands will bring up a dialog to rename the Stand Template.
\begin{itemize}
	\item - The other way users can do this is by double-clicking the Name cell of 
the selected Stand Template in the list of Stand Templates. This will cause that field to
become editable and upon changing the name and pressing 'Enter' the Stand Template will
be renamed.
\end{itemize} 

\emph{Delete Selected Stand} - Clicking the Delete Selected Stand button while a 
Stand Template is selected will remove the instance of the Stand from the Map Area.

\begin{itemize}
\item NOTE - This will not delete the Stand Template; only the selected instance
instance of the Stand on the mapping area.
\end{itemize}

\emph{Display Grid} - The Display Grid ToggleButton will turn on grid lines if users 
find it eases the Stand placement process. Clicking this button again will turn the grid
lines off.

\emph{Change Backdrop} - Clicking this button will bring up a file selection dialog 
allowing the user to select a .png file to display behind the mapping area as a 
helpful reference in where Stands should go in accordance to their garage.

\emph{Stand Metadata} - Below the list of Stand Templates, the user will find a
statusbar. This bar displays property information of the currently selected Stand
Template or the currently selected Stand in the mapping area. Information such as
the Stand Name, Height and Width and ID will be displayed.

\section{Frontend: Technical Design}
The entire Frontend portion of Map My Garage Sale was developed using Xamarin
Studio 5 on Windows. Xamarin Studio incorporates the Stetic UI designer for use with the GTK
Sharp library. Design of the User Interface was done with a combination of using the 
Stetic designer as well an manual instantiation and adjustments of UI Controls for 
increased flexibility. 
\subsection{User Interface}
GTK more or less resembles WinForms and thus suited to event-based programming.
Many of the features accessed in the User Interface are the result of control-specific 
events firing and the program reacting to them and making the necessary calls to the 
service layer. A Table was used as the base layout and inside the cells of the Table are 
VBox's and HBox's. The mapping area and Stand Templates NodeView are inserted
directly into Table in order to maintain proper spacing. 

Resizing is disabled in the User Interface due to the fact that directly impacted the 
performance of the mapping area since scaling the area up drastically increases 
overhead for redrawing the mapping area.

A NodeView and NodeStore were used in create the list of available Stand Templates.
A NodeView contains a NodeStore and each Node in the NodeStore can be considered 
an item in the box. Each icon consists of a PixBuf object that is in memory and is
disposed of when the application is closed.

Buttons that contain an icon in them are custom Widgets. They are a Box combined with a label 
and a passed in path to an image asset that exists in the subfolder 'Assets' in the root
folder of the application.

Non-generated code for User Interface initialization can be found in the SetupUI 
methods in the respective classes in the root Frontend namespace. This is code 
generated by the Stetic designer and it should not be manually modified.

\section{Mono}
\section{Cairo}
Cairo is a 2D graphics library that excels at drawing static graphics on drawables. 
It has solid support in the Mono community and is fully binded to GTK-Sharp. This made 
it a good choice over the System.Drawing library that Mono has implemented from 
Microsoft's .NET.

One additional option that was considered was binding the GooCanvas library to C-Sharp.
This would have allowed the use of a Canvas style mapping area and would have 
drastically reduced the amount of custom drawing functionality that needed to be 
implemented. However, there is currently no binding of GooCanvas to C-Sharp publically 
available and it would have put too much pressure on completing prior to using it in 
Map My Garage Sale.

\subsection{Map Design Area (Grid)}
Perhaps the most critical portion of the application is the actual mapping area. Cairo
and a GTK DrawingArea object were used to create the grid. Cairo uses a Context object 
to compose Strokes on a drawable. Since Cairo is not a Canvas based 2D graphics library, 
heavy use of the DrawingArea's Expose Event is maintained. 
Two objects exist to control what is draw to the mapping area. 

\subsection{Grid}
A static object holding properties such as the Height, Width, a backdrop image path and a bool 
for drawing grid lines leverages Cairo for drawing information pulled from the engine. The 
mapping area is drawn on a per Tile basis, where each Tile is 3 pixels. This Grid object loops 
through the engine's internal grid in memory and pulls back color information for each Tile. This 
coloris then applied to the drawable. The number of pixels per Tile was chosen based on getting the 
best possible performance for the grid. Having a 1:1 ratio would put too much load on the user's
machine for a drawing and the application's performance would decline to unacceptable levels.
This object draws the optional grid lines using the engine's internal grid dimensions and
creating a horizontal and vertical line every 50 pixels.

The use of Cairo's Surface object was used in order to draw the optional backdrop image. 
Cairo's Context object allows the setting of a Source such as a Cairo ImageSurface. 
This Surface object can load a file from a specified path and allow Cairo to 'Paint' it to 
a drawable. One caveat to this object is the image must be in a PNG file format in
order for Cairo to be able to properly create the image from the native, wrapped,
static method create_from_png that is called.

\subsection{Expose Event}
This Expose Event is fired when the DrawingArea needs to be redrawn, such as when the Widget
is shown or the event is queued up via a separate call. Map My Garage Sale takes advantage of
the ability to queue these calls up to refresh the mapping area with updated information from the 
service layer. In order to differentiate between what type of drawing should be done, a simple
enum is used with a switch in the Expose Event. In order to avoid having to redraw the whole
mapping area each time a change is made, a clipped region is created in some cases where the user
has recently interacted with the mapping area. This clipped region, specified by a combination
of the current mouse location and Height and Width boundaries allows only this area of the 
drawable to be redrawn and updated.  In most other cases however, performance is not impeded 
enough to warrant anything other than a full grid refresh.
\end{document}